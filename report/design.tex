% File: design.tex
% Date: Sun Sep 01 10:47:09 2013 +0800
% Author: Yuxin Wu <ppwwyyxxc@gmail.com>

\section{项目设计}
\subsection{目录结构}
项目的目录结构大致如下:
\dirtree{%
  .1 /.
  .2 report.pdf.
  .2 report/\DTcomment{本报告的\LaTeX 源代码}.
  .3 img/\DTcomment{报告中使用到的样例图片}.
  .2 demo/\DTcomment{一些样例}.
  .2 doc/\DTcomment{由Doxygen生成的程序文档}.
  .2 resource/\DTcomment{程序使用到的外部资源}.
  .2 simplified/\DTcomment{网格简化结果图,由main.cc中的函数批量生成}.
  .2 src/\DTcomment{程序源代码}.
  .3 include/\DTcomment{头文件}.
  .4 geometry/\DTcomment{实现抽象几何对象}.
  .4 renderable/\DTcomment{定义可渲染几何对象}.
  .4 lib/\DTcomment{程序使用的辅助函数定义}.
  .4 material/\DTcomment{实现表面属性及纹理}.
  .4 render/\DTcomment{定义图像渲染}.
  .3 lib/\DTcomment{辅助函数的实现}.
  .4 debugutils.cc.
  .4 utils.cc.
  .4 imagereader.cc.
  .4 Timer.cc.
  .3 renderable/\DTcomment{可渲染几何对象及其渲染相关操作的实现}.
  .4 face.cc.
  .4 mesh.cc.
  .4 plane.cc.
  .4 sphere.cc.
  .3 gui/\DTcomment{图形界面}.
  .4 main.cxx.
  .4 window.cxx, window.hh\DTcomment{主窗口相关操作实现}.
  .4 window.ui\DTcomment{UI定义}.
  .4 objviewer.pro\DTcomment{Qt工程文件}.
  .3 kdtree.cc.
  .3 cvrender.cc.
  .3 mesh\_simplifier.cc.
  .3 space.cc.
  .3 static\_const.cc.
  .3 view.cc.
  .3 main.cc.
  .3 Makefile.
  .3 Doxyfile.
}

\subsection{渲染物体相关类的设计}
所有可渲染物体,包括平面、球、面片、网格、KD树,均继承自\verb|Renderable|基类,当其需要与光线求交时,
通过\verb|Renderable::get_trace()|返回一个\verb|Trace|类的子类对象指针,由
\verb|Trace|类完成求交相关的操作.一个\verb|Trace|对象相当于一个物体与一条光线的组合.

\begin{figure}[H]
  \begin{minipage}[b]{0.46\linewidth}
    \centering
    \includegraphics[width=\textwidth]{res/renderable_inherit.png}
  \end{minipage}
  \begin{minipage}[b]{0.46\linewidth}
    \centering
    \includegraphics[width=\textwidth]{res/trace_inherit.png}
  \end{minipage}
\end{figure}

上图是\verb|Renderable|与\verb|Trace|的继承图,其中\verb|Light|继承自\verb|Renderable|是因为全局光照模型中
需要对光源求交,程序中只支持球形光源(在局部光照模型中将球形光源看做点光源).
\verb|Trace|仅有三个子类是由于\verb|Mesh|返回\verb|FaceTrace|对象指针, \verb|KDTree|返回它所管理物体对应的\verb|Trace|对象指针.

\verb|Renderable|类只有获取物体表面纹理及获取包围盒两种方法,
\verb|Trace|类包括了判断相交、求交点、求交点法向、交点表面属性、交点前方介质密度等方法.

这样设计的好处有如下两点:

\begin{enumerate}
    \item
      由\verb|Trace|对象自己管理求交过程的中间结果,
      保留有用的结果以备其他方法使用,节省了计算资源.
如判断是否相交时, 一些中间结果可能在计算交点距离时使用, 交点法向方向又有可能被计算纹理映射坐标时使用,
这些中间结果是每一对$ (物体,光线)$特有的,且不该对外界暴露,因而用\verb|Trace|类将其封装.

\item 这种设计使得同样的\verb|KDTree|类只需接受一个\verb|Renderable|对象的集合,
就可以很好的管理物体,实现了KD树的数据结构在网格以及在整个空间中的复用.
并且, \verb|KDTree|也是\verb|Renderable|子类, 因而能够支持KD树的嵌套.
程序中, 所有面片较多的网格对象内部用KD树管理面片,  同时空间中所有有限大物体也被KD树管理.

\end{enumerate}

\subsection{空间视图相关设计}
\begin{figure}[H]
  \centering
  \includegraphics[scale=0.4]{res/viewer_diagram.png}
\end{figure}
空间\verb|Space|类用于描述场景, 为一系列物体及光源的封装.

视图\verb|View|类为视点及屏幕的组合,负责生成光线对象,
并传入\verb|Space::trace()|获取相应颜色, 最终生成图像矩阵.

\verb|Viewer|类提供用户操作的接口,
根据用户操作调用\verb|View|类的一系列改变视角的方法\secref{navigate},
\verb|View|返回图像后,调用\verb|RenderBase|类进行显示.
