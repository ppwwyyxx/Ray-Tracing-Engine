% File: design.tex
% Date: Sat Jun 22 18:05:09 2013 +0800
% Author: Yuxin Wu <ppwwyyxxc@gmail.com>

\section{项目设计}
项目的目录结构大致如下:
\dirtree{%
  .1 /.
  .2 report/\DTcomment{本报告的\LaTeX 源代码}.
  .2 demo/\DTcomment{一些演示}.
  .2 resource/\DTcomment{程序使用到的外部资源}.
  .2 src/\DTcomment{程序源代码}.
  .3 include/\DTcomment{头文件}.
  .4 geometry/\DTcomment{实现抽象几何对象}.
  .4 renderable/\DTcomment{定义可渲染几何对象}.
  .4 lib/\DTcomment{程序使用的辅助函数定义}.
  .4 material/\DTcomment{实现表面属性及纹理}.
  .4 render/\DTcomment{定义图像渲染}.
  .3 lib/\DTcomment{辅助函数的实现}.
  .4 debugutils.cc.
  .4 utils.cc.
  .4 imagereader.cc.
  .4 Timer.cc.
  .3 renderable/\DTcomment{可渲染几何对象及其渲染相关操作的实现}.
  .4 face.cc.
  .4 mesh.cc.
  .4 plane.cc.
  .4 sphere.cc.
  .3 gui/\DTcomment{图形界面}.
  .4 main.cxx.
  .4 window.cxx.
  .4 window.hh.
  .4 window.ui.
  .4 objviewer.pro.
  .3 kdtree.cc.
  .3 cvrender.cc.
  .3 mesh\_simplifier.cc.
  .3 space.cc.
  .3 static\_const.cc.
  .3 view.cc.
  .3 main.cc.
  .3 Makefile.
  .3 Doxyfile.
}

\subsection{C++类设计}
\begin{enumerate}
  \item 渲染物体相关类的设计

    所有可渲染物体,包括平面、球、面片、网格、KD树,均继承自\verb|Renderable|基类,当其需要与光线求交时,
    通过\verb|Renderable::get_trace()|返回一个\verb|Trace|类的子类对象指针,由
    \verb|Trace|类完成求交相关的操作.一个\verb|Trace|对象相当于一个物体与一条光线的组合.

    \begin{figure}[H]
      \begin{minipage}[b]{0.46\linewidth}
        \centering
        \includegraphics[width=\textwidth]{res/renderable_inherit.png}
      \end{minipage}
      \begin{minipage}[b]{0.46\linewidth}
        \centering
        \includegraphics[width=\textwidth]{res/trace_inherit.png}
      \end{minipage}
    \end{figure}

    上图是\verb|Renderable|与\verb|Trace|的继承图,其中\verb|Trace|仅有三个子类是由于\verb|Mesh|返回\verb|FaceTrace|对象指针,
    \verb|KDTree|返回它所管理物体对应的\verb|Trace|对象指针.
    \verb|Renderable|类只有获取物体表面纹理及获取包围盒两种方法,
    \verb|Trace|类包括了判断相交、求交点、求交点法向、交点表面属性、交点前方介质密度等方法.

    这样做的好处是,由\verb|Trace|对象自己管理求交过程的中间结果,保留有用的结果以备其他方法使用,节省了计算资源.
    如判断是否相交时一些中间结果可能在计算交点距离时使用,交点法向方向有可能被计算纹理映射坐标时使用,
    这些中间结果是每一对$ (物体,光线)$特有的,又不该对外界暴露,因而用\verb|Trace|类将其封装.

    同时,这种设计使得同样的\verb|KDTree|类只需接受一个\verb|Renderable|对象的集合,
    就可以很好的管理物体,实现了KD树的数据结构在网格以及在整个空间中的复用,也能够支持KD树的嵌套.

  \item 空间视图相关设计
    \begin{figure}[H]
      \centering
      \includegraphics[scale=0.4]{res/viewer_diagram.png}
    \end{figure}
    空间\verb|Space|类为一系列物体及光源的封装.

    视图\verb|View|类为一个视点及屏幕的组合,负责生成光线并调用\verb|Space::trace()|获取相应颜色.

    \verb|Viewer|类提供用户操作的接口,
    根据用户操作调用\verb|View|类的一系列改变视角的方法\secref{navigate},同时调用\verb|RenderBase|类进行显示.

\end{enumerate}

\subsection{视图导航}
\label{sec:navigate}
命令行程序在演示时通过OpenCV的\verb|imshow()|函数输出图片,同时能够接收自定义的键盘事件,本程序支持的键盘操作如下:
\begin{table}[H]
\begin{tabular}{c|c}
  \shline
  \UArrow \DArrow& 屏幕以视点到其连线为轴旋转\\
  \LArrow \RArrow & 围绕固定中心旋转视点\\
  \keystroke{h}\keystroke{j}\keystroke{k}\keystroke{l} & 视点及屏幕的平移 \\
  \keystroke{=}\keystroke{-} & 缩放\\
  \keystroke{>}\keystroke{<} & 固定视点旋转视角\\
  \keystroke{]}\keystroke{[} & 调节焦平面远近(景深模式下有用)\\
  \keystroke{s} & 保存当前图片至output.png\\
  \keystroke{p} & 输出当前视角信息\\
  \keystroke{q} \Esc & 退出\\
\end{tabular}
  \centering
\end{table}



\subsection{图形界面}
